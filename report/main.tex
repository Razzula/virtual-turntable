%%%%%%%%%%%%%%%%%% USAGE INSTRUCTIONS %%%%%%%%%%%%%%%%%%
% - Compile using LuaLaTeX and biber, unless there is a particular reason not to. Do not use the older LaTex/PDFLaTeX or BibTeX. (The fonts won't work correctly.)
% - Font and the report 'year' must be specified when all \documentclass or the template won't work correctly. (There's no error checking/default cases!)
% - For best performance save images/graphics as PDF files, not as png/jpg/eps. This makes no difference to how images are inserted using \includegraphics.
% - As many further packages as wanted can be loaded. Below are just an example set. Note that template itself loads a number of packages, including hyperref.
% - References are handed using biblatex.
% - Link to the presentation of theses policy: https://documents.manchester.ac.uk/DocuInfo.aspx?DocID=2863



%%%%%%%%%%%%%%%%%% META DATA SETUP %%%%%%%%%%%%%%%%%%
% This is where the document title and author are set. Other details for the title page are set later
% Note that if/when you edit these you may need to 'Recompile from scratch' to get the changes to display in the PDF. (In Overleaf, select the down arrow to the right of the 'Recompile' button)
\begin{filecontents*}{\jobname.xmpdata}
  \Title{COMP30040 Report} 
  \Author{10826115} % should be student number rather than name to help with annoymous marking
  \Language{en-GB}
  \Copyrighted{True}
  % More meta-data fielda can be added here if wanted, see https://ctan.org/pkg/pdfx?lang=en for fields
\end{filecontents*}


%%%%%%%%%%%%%%%%%% DOCUMENT SETUP %%%%%%%%%%%%%%%%%%
\documentclass{uom_eee_dissertation_casson} 


%%%%%%%%%%%%%%%%%% PACKAGES AND COMMANDS %%%%%%%%%%%%%%%%%%

% Packages
\usepackage{graphicx,psfrag,color} % for postscript graphics files
  \graphicspath{ {./images/} }
\usepackage{amsmath}               % assumes amsmath package installed
  \allowdisplaybreaks[1]           % allow eqnarrays to break across pages
\usepackage{amssymb}               % assumes amsmath package installed 
\usepackage{url}                   % format hyperlinks correctly
\usepackage{rotating}              % allow portrait figures and tables
\usepackage{multirow}              % allows merging of rows in tables
\usepackage{lscape}                % allows pages to be typeset in landscape mode
\usepackage{tabularx}              % allows fixed width tables
\usepackage{verbatim}              % enhanced version of built-in verbatim environment
\usepackage{footnote}              % allows more control over footnote environments
\usepackage{float}                 % allows H option on floats to force here placement
\usepackage{booktabs}              % improve table line spacing
\usepackage{lipsum}                % for adding dummy text here
\usepackage[base]{babel}           % for proper hypthenation in lipsum sections
\usepackage{subcaption}            % for multiple sub-figures in a single float
% Add your packages here

% Optional: for adding alt-text to images:
%\usepackage{pdfcomment}            % for alt text for accessibility
% Then to add images use:
% \pdftooltip{\includegraphics[width=0.5\textwidth]{image.pdf}}{Alt-text here}
% This makes the text in the image non-select-able though (assuming it's a vector file)

% Custom commands
\newcommand{\degree}{\ensuremath{^\circ}}
\newcommand{\sus}[1]{$^{\mbox{\scriptsize #1}}$} % superscript in text (e.g. 1st)
\newcommand{\sub}[1]{$_{\mbox{\scriptsize #1}}$} % subscript in text
\newcommand{\sect}[1]{Section~\ref{#1}}
\newcommand{\fig}[1]{Fig.~\ref{#1}}
\newcommand{\tab}[1]{Table~\ref{#1}}
\newcommand{\equ}[1]{(\ref{#1})}
\newcommand{\appx}[1]{Appendix~\ref{#1}}



%%%%%%%%%%%%%%%%%% REFERENCES SETUP %%%%%%%%%%%%%%%%%%

% Setup your references here. Change the reference style here if wanted
\usepackage[style=ieee,backend=biber,backref=true,hyperref=auto]{biblatex}
% Note backref=true adds a page number (and hyperlink) to each reference so you can easily go back from the references to the main document. You may prefer backref=false if you need to stick strictly to a given reference style


% Fixes which can't be applied in the .cls file
\DefineBibliographyStrings{english}{backrefpage = {cited on p\adddot},  backrefpages = {cited on pp\adddot}}
%  \renewcommand*{\bibfont}{\large}


% Add more .bib files here if wanted
\addbibresource{references.bib}



%%%%%%%%%%%%%%%%%% START DOCUMENT %%%%%%%%%%%%%%%%%%

% Don't edit these lines, title and author are automatically taken from the document meta-data defined above
\begin{document}
\makeatletter
\title{\xmp@Title}
\studentid{\xmp@Author}
\makeatother

% Set the below yourself
\course{Computer Science}  % "Master of Science in" is added automatically
                                                   % Our courses are: Advanced Control and Systems Engineering, Advanced Control and Systems Engineering with Extended Research, Communications and Signal Processing, Communications and Signal Processing with Extended Research, Electrical Power Systems Engineering, Advanced Electrical Power Systems Engineering, Renewable Energy and Clean Technology, Renewable Energy and Clean Technology with Extended Research
\submitdate{2025}                                  % regulations ask only for the year, not month
\wordcount{TODO}		                           % use \wordcount{} to set the count, \thewordcount to print in the text
\maketitle


%%%%%%%%%%%%%%%%%% LISTS OF CONTENT %%%%%%%%%%%%%%%%%%
\uomtoc
% other lists are not required, but can include \uomlof and \uomlot if really want to

%%%%%%%%%%%%%%%%%% LISTS OF FIGURES %%%%%%%%%%%%%%%%%%%
\phantomsection\addcontentsline{toc}{section}{List of Figures}
  \section*{List of Figures}

%%%%%%%%%%%%%%%%%% LISTS OF TABLES %%%%%%%%%%%%%%%%%%%
\phantomsection\addcontentsline{toc}{section}{List of Tables}
  \section*{List of Tables}

%%%%%%%%%%%%%%%%%% ABBREVIATIONS %%%%%%%%%%%%%%%%%%%%%
\phantomsection\addcontentsline{toc}{section}{Abbreviations and Acronyms}
  \section*{Abbreviations and Acronyms}
    % ALWAYS define abbreviations on first use

    \begin{description}
      \item[API] Application Programming Interface
      \item[CNN] Convolutional Neural Network
      \item[OCR] Optical Character Recognition
    \end{description}

%%%%%%%%%%%%%%%%%% ABSTRACT %%%%%%%%%%%%%%%%%%%%%%%%%%
\begin{abstract} % put abstract here. Limit is 1 page.
  This is abstract text. 

\end{abstract}%
\clearpage



%%%%%%%%%%%%%%%%%% DECLARATIONS %%%%%%%%%%%%%%%%%%
\uomdeclarations % Don't need unless final thesis



%%%%%%%%%%%%%%%%%% ACKNOWLEDGEMENTS %%%%%%%%%%%%%%%%%%
\begin{uomacknowledgements}
No need to include, but can if want to.
\end{uomacknowledgements}



%%%%%%%%%%%%%%%%%% SECTION 1 %%%%%%%%%%%%%%%%%%
\section{Introduction}

  \subsection{Background and motivation}
  
  \subsection{Aims and objectives}

  % PROJECT PLAN?
  
  \subsection{Report structure} % ROADMAP


%%%%%%%%%%%%%%%%%% SECTION 2 %%%%%%%%%%%%%%%%%%
\section{Background and Literature Review} % LITERATURE REVIEW / BACKGROUND
    % summary of similar systems
    % explanations of concepts relied on later
    % advanatges and disadvantages of approaches
    % highlight problems I will improve
    % cite all references

    \subsection{Overview of Related Systems}
        \subsubsection{Vinyl Systems}
            Research into questions such as 'why do people like vinyl so much?' and 'why is it making a comeback?'. Traits distilled from this feed into later design choices (I have chosen to make a make a physical artefact with a mahogany base, because physicality and aesthetics are important the people who would potentially use this system).

        \subsubsection{Image Recognition}
            Papers that inform the use of different approaches, etc.

    \subsection{Legal and Ethical Considerations}
        % dataset collection


%%%%%%%%%%%%%%%%%% SECTION 3 %%%%%%%%%%%%%%%%%%
\section{Design} % DESIGN / METHODOLOGY
    What did past-Jack set out to do?
    % design diagrams!

    \subsection{Requirements Analysis}
        % list of requirements

    \subsection{System Architecture} % high-level system overview
        \subsubsection{Technology Stack}
            All of the technologies used (TS, React, bun, Python, FastAPI, etc.), and why them specifically.

        \subsubsection{Design Choices}
            % unique 1-1/1-many relation
            % server hardware control
            % host v. remote client
            Particular broad and niche design choices made, and why, such as:
            \begin{description}
                \item[1.] Why use a web approach for a localised device?
                \item[2.] Why use an unorthodox 1-1 Websocket approach for client-server calls?
            \end{description}
            Also things such as 'point of truth handling', etc.

    \subsection{Front-end}
        \subsubsection{User Interface}
            % lofi designs

            'No-UI' approach (or, more accurately, minimal UI)\\
            Physical user interaction controls

        \subsubsection{Audio Playback}
            % Spotify Playback SDK
            % Spotify Web API

        \subsubsection{Remote Clients}
            'Remote Control' UI from an external device (mobile, etc.)
	
	\subsection{Back-end}
        \subsubsection{Authentication}
            Might be redundant with the 'Security Considerations' information?

        \subsubsection{Metadata Retrieval}
            % Discogs API
            % Discogs ToS (not used for AI) 
        
        \subsubsection{Hardware Interaction}
            % RPi
	
	\subsection{Machine Learning Model Design}
        \subsubsection{Dataset Collection}
            % use of CoverArtArchive

        \subsubsection{Model Architecture}

    \subsection{Security Considerations}
        % handling of auth tokens (transient)
        % off-handing of persistence tasks
        % network hosting
        % option of same-user or any-user controls
        % API compliances

    \subsection{Testing Methodology}
        Design of tests and evaluations; plan for unit testing, model evaluation, etc.


%%%%%%%%%%%%%%%%%% SECTION 4 %%%%%%%%%%%%%%%%%%
\section{Implementation}
    Details realised in practice, decisions made, etc.\\
    Challenges encountered, how addressed, etc.
    % details realised in practice
    % challenges, and how overcame
    % code-level or system-specific decisions, optimizations, or trade-offs
    % integration
    \subsection{Front-end}
        \subsubsection{Challenges Encountered}

    \subsection{Back-end}
        \subsubsection{Challenges Encountered}
            Maybe the power-cable low-voltage issues could be mentioned? Not sure how much there is to say...

    \subsection{Machine Learning Model}
        \subsubsection{System Integration}
            Practice-driven discoveries, such as 'how much is not too much' when it came to model architecture and the Pi's specs.

        \subsubsection{Challenges Encountered}
            Internet Archive downtime!


%%%%%%%%%%%%%%%%%% SECTION 5 %%%%%%%%%%%%%%%%%%
\section{Results} % edit section heading as appropriate
    What was actually produced?
    % results of soft/hardware testing
    % screenshots of UI / program output
    \subsection{Software Artefact}

    \subsection{Hardware Artefact}

%%%%%%%%%%%%%%%%%% SECTION 6 %%%%%%%%%%%%%%%%%%
\section{Evaluation}
    % does it do what it is supposed to do?
    % how well?
    % how well against others?
    \subsection{Machine Learning Model Performance}

    \subsection{User Experience}

    \subsection{Comparison with Existing Systems}

    \subsection{Ethical Implications}


%%%%%%%%%%%%%%%%%% SECTION 7 %%%%%%%%%%%%%%%%%%
\section{Conclusions and future work} % edit section heading as appropriate
    \subsection{Conclusions}
        % summarise results
        % achieved aims?
        % improvements!
	
	\subsection{Future work}
        % ideas for further work
        % big ideas; what could be done with my project?


%%%%%%%%%%%%%%%%%% REFERENCES %%%%%%%%%%%%%%%%%%
%\clearpage % uncomment to start on a new page if wanted
\printbibliography[title={References},heading=bibintoc] % a single list of references for the whole thesis



%%%%%%%%%%%%%%%%%% APPENDICES %%%%%%%%%%%%%%%%%%
\begin{uomappendix} 
    % screen dumps of UI
    % important but large results
    \section{Project outline}
    Project outline as submitted at the start of the project is a required appendix. Put here. 
    
    \section{Risk assessment}
    Risk assessment is a required appendix. Put here.
    
    %\section{Other appendices as necessary}
\end{uomappendix}


%%%%%%%%%%%%%%%%%% END MATTER %%%%%%%%%%%%%%%%%%
\end{document}